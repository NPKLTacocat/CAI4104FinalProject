\documentclass[10pt]{article}

\usepackage[left=0.8in,right=0.8in,top=0.15in,bottom=0.8in]{geometry}
\usepackage{xcolor}
\usepackage{hyperref}

\hypersetup{colorlinks=true,linkcolor=blue,urlcolor=blue}
\urlstyle{rm}
\usepackage{url}


\title{CAI 4104: Machine Learning Engineering\\
	\large Project Report:  {\textcolor{purple}{Title of the Project}}} %% TODO: replace with the title of your project
	
	
	
%% TODO: your name and email go here (all members of the group)
%% Comment out as needed and designate a point of contact
%% Add / remove names as necessary
\author{
        Your Name \\{\em (Point of Contact)} \\
        email1@ufl.edu\\
        \and
        Member 2's Name \\
        email2@ufl.edu\\
        \and
        Member 3's Name \\
        email3@ufl.edu\\
        \and
        Member 4's Name \\
        email4@ufl.edu\\
        \and
        Member 5's Name \\
        email5@ufl.edu\\
}


% set the date to today
\date{\today}


\begin{document} % start document tag

\maketitle



%%% Remember: writing counts! (try to be clear and concise.)
%%% Make sure to cite your sources and references (use refs.bib and \cite{} or \footnote{} for URLs).
%%%



%% TODO: write an introduction to make the report self-contained
%% Must address:
%% - What is the project about and what is notable about the approach and results?
%%
\section{Introduction}

% TODO:
Given a dataset containing 128 x 128 RGB PNG images containing one of 12 different objects: backpack, book, calculator, chair, clock, desk, keychain, laptop, paper, pen, phone, water bottle. The
goal of this project is to correctly classify images into one of the 12 classes mentioned above. The dataset is split into 70\% training, 15\% validation and 15\% testing. We will use a 
convolutional neural network (CNN) to classify the images. The CNN will consist of several convolutional layers, followed by max pooling layers, and finally fully connected layers. 
We will use the Adam optimizer and categorical cross-entropy loss function to train the model. The model will be evaluated using accuracy.



%% TODO: write about your approach / ML pipeline
%% Must contain:
%% - How you are trying to solve this problem
%% - How did you process the data?
%% - What is the task and approach (ML techniques)?
%%
\section{Approach}

\subsection{Data Preprocessing}

Our data preprocessing pipeline consists of several key steps to prepare the images for training:

\begin{itemize}
    \item \textbf{Image Resizing:} Although it is assumed that all the images are 128x128 pixels, we still resized all the images to 128x128 pixels to ensure uniformity.
    and that the expected batch size match the input size of the model.
    
    \item \textbf{Data Augmentation:} Additionally since the given dataset only consists of 4757 images, which is relatively small, we applied data augmentation to just the 
    training set to artificially increase training diversity. Furthermore, this also act as a regularization technique to prevent 
    overfitting which we later encountered during training. The data augmentation techniques we used include:
    \begin{itemize}
        \item Random horizontal flips
        \item Random vertical flips
    \end{itemize}
    These techiques should help the model generalize better to unseen data as the objects might appear in different orientations.
    \item \textbf{Normalization:} The input were first applied with transform.ToTensor which would have perfomed scaling on the pixel values from the range of [0, 255] to [0, 1]. 
    All images were then normalized using mean = [0.5, 0.5, 0.5] and standard deviation = [0.5, 0.5, 0.5] for each color channel to center the data around 0 and remap it to a range of 
    [-1, 1] and ensure consistent scale.
    
    \item \textbf{Data Splitting:} We divided the dataset into three subsets:
    \begin{itemize}
        \item Training set: 70\% of the data
        \item Validation set: 15\% of the data
        \item Test set: 15\% of the data
    \end{itemize}
    The split was performed with stratification to maintain class distribution across splits.
\end{itemize}


% TODO:
Write here. 





%% TODO: write about your evaluation methodology
%% Must contain:
%% - What are the metrics?
%% - What are the baselines?
%% - How did you split the data?
%%
\section{Evaluation Methodology}

% TODO:
Write here. 




%% TODO: write about your results/findings
%% Must contain:
%% - results comparing your approach to baseline according to metrics
%%
%% You can present this information in whatever way you want but consider tables and (or) figures.
%%
\section{Results}

% TODO:
Write here. 





%% TODO: write about what you conclude. This is not meant to be a summary section but more of a takeaways/future work section.
%% Must contain:
%% - Any concrete takeaways or conclusions from your experiments/results/project
%%
%% You can also discuss limitations here if you want.
%%
\section{Conclusions}

% TODO:
Write here. 

%%%%

\bibliography{refs}
\bibliographystyle{plain}


\end{document} % end tag of the document
